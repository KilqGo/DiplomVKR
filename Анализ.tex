\section{Анализ предметной области}
\subsection{Центры сервисного обслуживания персональных компьютеров}

Сейчас сложно представит современный мир без персональных компьютеров, включая стационарные системы, ноутбуки, моноблоки, сопутствующую периферию. Это универсальный инструмент работы, обучения, коммуникации и развлечения, который упрощает множество процессов. Однако как и любое технически сложное устройство, он подлежит износу, поломкам, программным сбоям и требуют периодического технического обслуживания для продления срока эксплуатации. Сервисные центры занимаются этим самым обслуживанием техники, имея под своим крылом квалифицированных инженеров, обеспечивают беспрерывную и эффективную работу компьютерной техники, как частных лиц, так и организаций. 

Обычно сервисные центры заняты диагностикой и устранением неисправностей систем, заменой вышедших из строя компонентов, ремонтом этих самых компонентов, устранением программных ошибок, восстановлением данных с поврежденных носителей. В сферу услуг сервисных центров также входит аппаратная чистка системных блоков и ноутбуков от пыли, замена расходных материалов, обновление и оптимизация программной части системы, возможна установка операционной системы или другого программного обеспечения по желанию пользователя, которому могут также понадобиться консультации или улучшение аппаратной части персонального компьютера. Сервисные центры часто выступают авторизованными партнерами производителей оборудования, выполняя гарантийный ремонт и постгарантийное обслуживание.

В следствии работы сервис центров пользователи экономят деньги, время, получают качественный ремонт и обслуживание своей дорогостоящей техники, а производители получают каналы выполнения гарантийного ремонта, повышение лояльности клиентов за качественное обслуживание и сбор информации по типовым неисправностям для устранения в последующем продукте.

Для полноценной работы сервис-центра требуется наличие квалифицированных специалистов по ремонту, системных администраторов и консультантов, специализированного оборудования и инструментов для диагностики и ремонта, эффективное управление складом оригинальных и совместимых комплектующих, расходных материалов для оперативного выполнения ремонтов. По хорошему стоит обеспечить четкие процедуры приема заказа, диагностики, согласование стоимости и сроков, контроля ремонта и выдачи гарантийного сопровождения, а также прозрачность процессов, удобство коммуникации и возможность выезда мастера на место. 	

\subsection{Характеристика сервис-центра PC-Club}

Сервис-центр «PC-Club» позиционирует себя как специализированное предприятие, оказывающее полный спектр услуг по ремонту, обслуживанию, модернизации и поддержке персональных компьютеров и периферийных устройств. Сами же работы проводятся как с частными пользователями, так и с корпоративными или юридическими лицами, которым нужна стабильная и бесперебойная работа своей ИТ-инфраструктуры.

Сервисный-центр готов работать с моделями всех основных производителей ноутбуков, системными блоками от известных брендов или кастомными сборками пользователей, периферийными устройствами(мониторы, принтеры/мфу, сканеры, клавиатуры, мыши, сетевое оборудование, внешние накопители).

Список услуг сервис-центра «PC-Club» довольно широк, например для ноутбуков предоставляются следующие варианты аппаратного ремонта: замена или если возможно ремонт матрицы, клавиатуры, корпусных деталей, аккумуляторных батарей, разъема питания, материнской платы, жесткого диска, оперативной памяти, цепей питания, тачпада, чистка системы охлаждения, корпуса и замена расходных материалов. Для системных же блоков «PC-Club» предоставляет: замену или если возможно ремонт материнской платы, процессора, видеокарты, блока питания, оперативной памяти, системы охлаждения, жестких дисков или твердотельных накопителей. За периферийные устройства сервис-центр берется с не меньшей охотой и ремонтирует платы управления принтеров или МФУ, блоки питания мониторов, заменяет ролики подачи бумаги, ламп и матриц в проекторах. Не стоит забывать, что кроме починки аппаратной части «PC-Club» может проводить глубокую диагностику компьютеров и периферии, установить операционную систему, почистить систему от вирусов, шпионского ПО, майнеров, обновить программные компоненты и проконсультировать пользователей по любым вопросам. Часть этих услуг возможно предоставлять с выездом мастера к клиенту, где тот может исправить на месте или отправить устройство в мастерскую.

В первую очередь успешная работа «PC-Club» обеспечивается персоналом, состоящим из квалифицированных специалистов, таких как: инженеры по ремонту, отвечающие за ремонт компонентов с опытом в микроэлектронике; системные администраторы, специалисты по программному обеспечению и операционным системам; менеджеры по продажам и работе с клиентами.

В мастерской сервис центра есть оборудования для выполнения любого рода ремонтных работ, чистки, диагностические и тестовые стенды. Без грамотной организации склада не получится эффективно и быстро выполнять заказы клиентов, для этого нужен точный количественный учет, адресное хранение и регулярный анализ, закупка недостающих позиций.

Гарантийная политика «PC-Club» максимально удобная для клиентов, а именно от 3 до 12 месяцев гарантии на все работы. Гарантия распространяется на устранение недостатков ремонта, случаи выхода из строя комплектующих при соблюдении клиентом правил эксплуатации.

\subsection{Необходимость цифровизации работы сервис-центра}

Сервис-центр «PC-Club» сложная организация с большим количеством услуг и разнообразным наполнением склада запчастей, ей необходима координация работы своих специалистов и трекинг заказов, хранение их в цифровом виде. Цифровизация работы сервис-центра упростит и организует множество процессов, что необходимо для выживания в условиях конкуренции. Основная цель внедрения цифровых инструментов - не просто привлечение новых клиентов, а фундаментальное повышение эффективности управления внутренними ресурсами и процессами. Автоматизация и организация большей части процессов есть ключ в достижении повышенной скорости ремонта, минимизации ошибок и потерь данных бумажных носителей, оптимизации затрат и улучшение качества, скорости сервиса, что в конечном счете приведет к росту удовлетворенности существующих клиентов и привлечению новых. Цифровизация – это инвестиция в операционное превосходство и устойчивость бизнеса.

\subsection{Анализ существующих программных продуктов для сервис-центров}

RemOnline:

Программа предназначена для комплексного управления услугами (включая, но не ограничиваясь ремонтом) в облачной среде.

Достоинствами программы являются: высокая скорость работы; обширная библиотека готовых шаблонов документов; эффективная и оперативная техническая поддержка; постоянное развитие функциональных возможностей; наличие полноценного складского учета; наличие мобильного приложения.
Недостатками являются: отсутствие десктопной версии (исключительно облачное решение); блокировка доступа к данным пользователя при неоплате подписки; тарификация услуг в евро; высокая стоимость тарифных планов.

HelloClient:

Программа предназначена для облачного управления клиентскими отношениями и бизнес-процессами с акцентом на простоту использования.

Достоинствами программы являются: интуитивно понятный и простой интерфейс; доступная ценовая политика без ограничений по количеству сотрудников; наличие функционального мобильного приложения; возможность интеграции с онлайн-кассами; наличие инструмента для импорта данных из системы Ремонлайн.
Недостатками являются: закрытый API, ограничивающий возможности сторонних интеграций; отсутствие учета товарно-материальных ценностей по серийным номерам (что критично для управления компонентами).

Gincore:

Программа предназначена для комплексного учета бизнес-процессов с фокусом на бухгалтерские операции и складской менеджмент.

Достоинствами программы являются: мощный функционал для глубокого бухгалтерского и финансового учета (учет расчетов, прибыли); гибкая система расчета заработной платы на основе KPI; поддержка детального учета ТМЦ по серийным номерам; развитые механизмы контроля деятельности сотрудников.
Недостатками являются: высокая сложность освоения и значительные временные затраты на первоначальную настройку; отсутствие мобильного приложения; отсутствие встроенной интеграции с онлайн-кассами.

LiveSklad:

Программа предназначена для облачного управления заказами, складскими операциями и аналитикой, являясь функциональным аналогом Ремонлайн.

Достоинствами программы являются: схожесть интерфейса и функциональных возможностей с Ремонлайн при более низкой стоимости; эффективный контроль сроков выполнения заказов; гибкие схемы расчета заработной платы персонала; поддержка работы с онлайн-кассами и штрих-кодированием; наличие развитых инструментов аналитики.
Недостатками являются: отсутствие мобильного приложения; неудобство использования шаблонов договоров; отсутствие возможности синхронизации данных с системами 1С.

Вулкан-М:

Программа предназначена для базового онлайн-учета клиентской базы и операционной деятельности с фиксированной стоимостью подписки.

Достоинствами программы являются: предельно простой и минималистичный интерфейс; фиксированная стоимость подписки без скрытых платежей; отсутствие ограничений на количество пользователей (сотрудников).
Недостатками являются: низкая скорость реакции и долгие сроки решения вопросов технической поддержкой; признаки возможного прекращения активной разработки и обновления продукта; отсутствие реально работающих инструментов для мониторинга KPI персонала.

ServiceApp:

Программа предназначена для контроля бизнес-процессов, финансовых потоков и складских операций в сервисных центрах.

Достоинствами программы являются: удобный и понятный пользовательский интерфейс; детальный контроль этапов выполнения каждого заказа; расширенные возможности ведения и анализа клиентской базы; гибкая система настройки шаблонов документов (акты, накладные, ценники); удобные инструменты для комплексного финансового учета и расчета заработной платы.
Недостатками являются: отсутствие специализированного мобильного приложения для мастеров, что затрудняет оперативную фиксацию этапов ремонта и доступ к актуальным складским данным; недостаточно развитый функционал интеграций.

REMDESK:

Программа предназначена для специализированного управления деятельностью сервисных центров и ремонтных мастерских.

Достоинствами программы являются: ориентация функционала на специфику ремонтного бизнеса; детальный контроль работы мастеров по каждому этапу ремонта; поддержка раздельных кассовых узлов для работы с несколькими филиалами; наличие глубокой финансовой аналитики и статистики.
Недостатками являются: модель тарификации, основанная на количестве мастеров; отсутствие мобильного приложения; отсутствие встроенных интеграций с популярными онлайн-кассами и маркетплейсами.




