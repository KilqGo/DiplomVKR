\section{Анализ предметной области}
\subsection{Центры сервисного обслуживания персональных компьютеров}

Сейчас сложно представит современный мир без персональных компьютеров, включая стационарные системы, ноутбуки, моноблоки, сопутствующую периферию. Это универсальный инструмент работы, обучения, коммуникации и развлечения, который упрощает множество процессов. Однако как и любое технически сложное устройство, он подлежит износу, поломкам, программным сбоям и требуют периодического технического обслуживания для продления срока эксплуатации. Сервисные центры занимаются этим самым обслуживанием техники, имея под своим крылом квалифицированных инженеров, обеспечивают беспрерывную и эффективную работу компьютерной техники, как частных лиц, так и организаций. 

Обычно сервисные центры заняты диагностикой и устранением неисправностей систем, заменой вышедших из строя компонентов, ремонтом этих самых компонентов, устранением программных ошибок, восстановлением данных с поврежденных носителей. В сферу услуг сервисных центров также входит аппаратная чистка системных блоков и ноутбуков от пыли, замена расходных материалов, обновление и оптимизация программной части системы, возможна установка операционной системы или другого программного обеспечения по желанию пользователя, которому могут также понадобиться консультации или улучшение аппаратной части персонального компьютера. Сервисные центры часто выступают авторизованными партнерами производителей оборудования, выполняя гарантийный ремонт и постгарантийное обслуживание.

В следствии работы сервис центров пользователи экономят деньги, время, получают качественный ремонт и обслуживание своей дорогостоящей техники, а производители получают каналы выполнения гарантийного ремонта, повышение лояльности клиентов за качественное обслуживание и сбор информации по типовым неисправностям для устранения в последующем продукте.

Для полноценной работы сервис-центра требуется наличие квалифицированных специалистов по ремонту, системных администраторов и консультантов, специализированного оборудования и инструментов для диагностики и ремонта, эффективное управление складом оригинальных и совместимых комплектующих, расходных материалов для оперативного выполнения ремонтов. По хорошему стоит обеспечить четкие процедуры приема заказа, диагностики, согласование стоимости и сроков, контроля ремонта и выдачи гарантийного сопровождения, а также прозрачность процессов, удобство коммуникации и возможность выезда мастера на место. 	

\subsection{Характеристика сервис-центра PC-Club}

Сервис-центр «PC-Club» позиционирует себя как специализированное предприятие, оказывающее полный спектр услуг по ремонту, обслуживанию, модернизации и поддержке персональных компьютеров и периферийных устройств. Сами же работы проводятся как с частными пользователями, так и с корпоративными или юридическими лицами, которым нужна стабильная и бесперебойная работа своей ИТ-инфраструктуры.

Сервисный-центр готов работать с моделями всех основных производителей ноутбуков, системными блоками от известных брендов или кастомными сборками пользователей, периферийными устройствами(мониторы, принтеры/мфу, сканеры, клавиатуры, мыши, сетевое оборудование, внешние накопители).

Список услуг сервис-центра «PC-Club» довольно широк, например для ноутбуков предоставляются следующие варианты аппаратного ремонта: замена или если возможно ремонт матрицы, клавиатуры, корпусных деталей, аккумуляторных батарей, разъема питания, материнской платы, жесткого диска, оперативной памяти, цепей питания, тачпада, чистка системы охлаждения, корпуса и замена расходных материалов. Для системных же блоков «PC-Club» предоставляет: замену или если возможно ремонт материнской платы, процессора, видеокарты, блока питания, оперативной памяти, системы охлаждения, жестких дисков или твердотельных накопителей. За периферийные устройства сервис-центр берется с не меньшей охотой и ремонтирует платы управления принтеров или МФУ, блоки питания мониторов, заменяет ролики подачи бумаги, ламп и матриц в проекторах. Не стоит забывать, что кроме починки аппаратной части «PC-Club» может проводить глубокую диагностику компьютеров и периферии, установить операционную систему, почистить систему от вирусов, шпионского ПО, майнеров, обновить программные компоненты и проконсультировать пользователей по любым вопросам. Часть этих услуг возможно предоставлять с выездом мастера к клиенту, где тот может исправить на месте или отправить устройство в мастерскую.

В первую очередь успешная работа «PC-Club» обеспечивается персоналом, состоящим из квалифицированных специалистов, таких как: инженеры по ремонту, отвечающие за ремонт компонентов с опытом в микроэлектронике; системные администраторы, специалисты по программному обеспечению и операционным системам; менеджеры по продажам и работе с клиентами.

В мастерской сервис центра есть оборудования для выполнения любого рода ремонтных работ, чистки, диагностические и тестовые стенды. Без грамотной организации склада не получится эффективно и быстро выполнять заказы клиентов, для этого нужен точный количественный учет, адресное хранение и регулярный анализ, закупка недостающих позиций.

Гарантийная политика «PC-Club» максимально удобная для клиентов, а именно от 3 до 12 месяцев гарантии на все работы. Гарантия распространяется на устранение недостатков ремонта, случаи выхода из строя комплектующих при соблюдении клиентом правил эксплуатации.

\subsection{Необходимость цифровизации работы сервис-центра}

Сервис-центр «PC-Club» сложная организация с большим количеством услуг и разнообразным наполнением склада запчастей, ей необходима координация работы своих специалистов и трекинг заказов, хранение их в цифровом виде. Цифровизация работы сервис-центра упростит и организует множество процессов, что необходимо для выживания в условиях конкуренции. Основная цель внедрения цифровых инструментов - не просто привлечение новых клиентов, а фундаментальное повышение эффективности управления внутренними ресурсами и процессами. Автоматизация и организация большей части процессов есть ключ в достижении повышенной скорости ремонта, минимизации ошибок и потерь данных бумажных носителей, оптимизации затрат и улучшение качества, скорости сервиса, что в конечном счете приведет к росту удовлетворенности существующих клиентов и привлечению новых. Цифровизация – это инвестиция в операционное превосходство и устойчивость бизнеса.

\subsection{Анализ существующих программных продуктов для сервис-центров}

Ремонлайн (RemOnline):
\begin{enumerate}
	\item Суть: Универсальная облачная CRM для услуг (не только ремонт). Огромный функционал: заказы, клиенты, склад, документы, email/SMS, отчеты, KPI, интеграции.
	\item Плюсы: Быстрый, много шаблонов, отличная поддержка, развивается, полноценный складской учет, мобильное приложение.
	\item Минусы: Только облако (нет десктоп), данные блокируются при неоплате, цены в евро, высокая стоимость тарифов.
	\item Для PC-Club: Мощный, но дорогой. Подойдет, если бюджет позволяет и нужен максимум функций.
\end{enumerate}

HelloClient:
\begin{enumerate}
	\item Суть: Облачная CRM, акцент на простоте и управлении процессами. База клиентов, задачи, воронки, каталог, история, импорт/экспорт, колл-центр.
	\item Плюсы: Очень простая, доступная цена без лимита сотрудников, мобильное приложение, интеграция с кассой, импорт из Ремонлайн.
	\item Минусы: Закрытый API, нет учета по серийным номерам, что критично для запчастей.
	\item Для PC-Club: Хорош для старта и управления клиентами/процессами, но слаб для детального учета запчастей.
\end{enumerate}

Gincore:
\begin{enumerate}
	\item Суть: Комплексное решение с упором на бухгалтерию и склад. Учет расчетов, прибыли, зарплаты (KPI), склад с серийниками, аналитика, контроль мастерской.
	\item Плюсы: Очень мощный, глубокая бухгалтерия, учет по серийникам, контроль сотрудников, гибкая зарплата.
	\item Минусы: Сложный в освоении и настройке, нет мобильного приложения, нет интеграции с онлайн-кассой.
	\item Для PC-Club: Максимум контроля, но требует времени на внедрение. Нет кассы и мобильности — минус.
\end{enumerate}

LiveSklad:
\begin{enumerate}
	\item Суть: Облачная CRM, аналог Ремонлайн по интерфейсу и функциям. Заказы, сроки, SMS, зарплата, качество, кассы, штрих-коды, склад, аналитика.
	\item Плюсы: Похож на Ремонлайн, дешевле Ремонлайн, контроль заказов, гибкая зарплата, касса, статистика.
	\item Минусы: Нет мобильного приложения, неудобные шаблоны договоров, нет синхронизации с 1С.
	\item Для PC-Club: Хороший баланс функционала и цены, альтернатива Ремонлайн. Отсутствие мобильного приложения — ограничение.
\end{enumerate}

Вулкан-М:
\begin{enumerate}
	\item Суть: Простое онлайн-решение. База клиентов, персонал, история, отчеты, интеграции. Фиксированная цена без лимита сотрудников.
	\item Плюсы: Очень простая, фиксированная цена, неограниченные сотрудники.
	\item Минусы: Долгая техподдержка, возможно прекратил развитие, нет реального мониторинга KPI персонала.
	\item Для PC-Club: Самый бюджетный вариант "на базовые нужды". Рискован из-за возможного отсутствия развития и поддержки.
\end{enumerate}

ServiceApp:
\begin{enumerate}
	\item Суть: Удобная программа для контроля процессов и финансов. Гибкие акты приема, комплектующие, склады, финансы, связи документов, ценники и штрих-коды, зарплата.
	\item Плюсы: Понятный интерфейс, контроль этапов заказа, детальная база клиентов, гибкие шаблоны документов, удобный финансовый учет, расчет зарплаты.
	\item Минусы: отсутствие мобильного приложения для мастеров может замедлять оперативную фиксацию этапов ремонта и доступ к данным склада.
	\item Для PC-Club: Сильный акцент на удобстве работы с документами, финансами и складом. Однако недостаточно функционала. 
\end{enumerate}

REMDESK:
\begin{enumerate}
	\item Суть: Специализированная программа именно для сервисных центров и мастерских. Организация работы сервиса и мастеров, документы и заявки, реальная аналитика. Тариф за мастеров.
	\item Плюсы: Фокус на ремонте, контроль мастеров по этапам, раздельные кассы для филиалов, финансовая статистика.
	\item Минусы: Цена за сотрудников, нет мобильного приложения, отсутствие интеграций с кассами, маркетплейсами.
	\item Для PC-Club: Хорошо заточен под управление ремонтами и мастерами. Отсутствие мобильности и интеграций — серьезное ограничение для роста.
\end{enumerate}









