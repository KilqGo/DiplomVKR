\section{Анализ предметной области}
\subsection{Характеристика сервис-центра и его деятельности.}

Историю развития ЭВМ можно разделить на пять поколений, ближе к концу войны появлялись первые вычислительные машины "первого поколения", основой в них служили электронные лампы. Быстродействие таких машин оставляло желать лучшего, а габариты и потребление энергии были огромными, ко всему прочему они не отличались высокой надежностью. Взаимодействие с таким компьютером сводилось к написанию машинного кода, это крайне трудоемкий процесс, требующий от пользователя высоких знаний работы с самим устройством машины, к таким машинам относятся ENIAC и UNIVAC.

Электронные лампы оказались не самым удобным решением в построении базы ЭВМ, важным событием стал переход на полупроводниковые элементы(транзисторы). Транзисторные ЭВМ относят к второму поколению, с каждым поколением преследовались похожие цели, уменьшение габаритов, энергопотребления, увеличение вычислительной мощности.

Развитие компьютерных технологий продолжается и по сей день. Они становятся доступны всё большему количеству простых пользователей, обрастают огромным количеством функций и возможностей. Кто поможет обычным потребителям собрать подходящий компьютер для любых нужд, а также поможет в случае каких либо неисправностей или вопросов?

Основная цель сервис-центра PC--Club сборка уникальных, качественных и индивидуальных системных блоков для каждого из своих клиентов. А также предоставление всевозможных услуг в обслуживание персональных компьютеров и периферии. 
\subsection{Веб-приложение}

Для организации эффективной работы сервис-центра требуется централизовать хранение информации, реализовать удобный интерфейс взаимодействия сотрудника или пользователя с системой, разграничить правила для пользователей, добавить возможность формирования актов приёма и выполнения работ. Скорее всего потребуется редактировать количество и виды компонентов имеющихся на складе, возможно заказывать новые. Для удобства можно реализовать все эти функции через веб-приложение в котором:

Пользователь
\begin{itemize}
	\item Сможет найти контакты сервиса;
	\item Оформить заказ без лишнего вмешательства;
\end{itemize}

Сотрудник
\begin{itemize}
	\item Может оформить заказ вручную если потребуется клиенту;
	\item Может редактировать данные о компонентах;
	\item Получит возможность заказывать недостающие комплектующие;
\end{itemize}
