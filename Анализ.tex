\section{Анализ предметной области}
\subsection{Эволюция ЭВМ}

Историю развития ЭВМ можно разделить на пять поколений, ближе к концу войны появлялись первые вычислительные машины "первого поколения", основой в них служили электронные лампы. Быстродействие таких машин оставляло желать лучшего, а габариты и потребление энергии были огромными, ко всему прочему они не отличались высокой надежностью. Взаимодействие с таким компьютером сводилось к написанию машинного кода, это крайне трудоемкий процесс, требующий от пользователя высоких знаний работы с самим устройством машины, к таким машинам относятся ENIAC и UNIVAC.

Электронные лампы оказались не самым удобным решением в построении базы ЭВМ, важным событием стал переход на полупроводниковые элементы(транзисторы). Транзисторные ЭВМ относят к второму поколению, с каждым поколением преследовались похожие цели, уменьшение габаритов, энергопотребления, увеличение вычислительной мощности.

Развитие компьютерных технологий продолжается и по сей день. Они становятся доступны всё большему количеству простых пользователей, обрастают огромным количеством функций и возможностей. Кто поможет обычным потребителям собрать подходящий компьютер для любых нужд, а также поможет в случае каких либо неисправностей или вопросов?

\subsubsection{необходимость автоматизации сервис-центра}
Основная цель сервис-центра PC--Club сборка уникальных, качественных и индивидуальных системных блоков для каждого из своих клиентов. А также предоставление всевозможных услуг в обслуживание персональных компьютеров и периферии. 

Для организации эффективной работы сервис-центра требуется централизовать хранение информации, реализовать удобный интерфейс взаимодействия сотрудника или пользователя с системой, разграничить правила для пользователей, добавить возможность формирования актов приёма и выполнения работ. Скорее всего потребуется редактировать количество и виды компонентов имеющихся на складе, возможно заказывать новые.

\subsection{Характеристика сервис-центра и его деятельности}
На современном рынке IT повышается спрос на гибкость в выборе комплектующих и сборке техники, пользователи все чаще предпочитают индивидуальные сборки готовым решениям. Однако процесс подбора комплектующих их совместимости и сборки остается сложным для неподготовленного пользователя

Сборка ПК под индивидуальные требования, включая:
\begin{itemize}
	\item консультации по выбору компонентов (процессор, видеокарта, охлаждение и т.д.);
	\item тестирование совместимости и стресс-тесты системы.
\end{itemize}

Обслуживание и ремонт, в том числе:
\begin{itemize}
	\item улучшение существующих систем;
	\item диагностика и устранение неисправностей;
	\item чистка, замена термопасты, настройка ПО.
\end{itemize}

\subsection{Анализ аналогичных решений}
На рынке присутствуют как онлайн-платформы (например, DNS Configurator, CyberPowerPC), так и локальные сервис-центры. Их слабые стороны:

Онлайн-конфигураторы:
\begin{itemize}
	\item ограниченная поддержка пользователей на этапе выбора;
	\item отсутствие интеграции с услугами постгарантийного обслуживания.
\end{itemize}

Локальные сервисы:
\begin{itemize}
	\item устаревшие системы учёта заказов (часто на основе Excel или бумажных журналов);
	\item низкая скорость обработки запросов из-за ручного ввода данных.
\end{itemize}

Какие преимущества у Pc-Club по сравнению с ними? Персональный подход к каждому клиенту, перенос в цифровое пространство части процессов и автоматизация. Внедрение веб-приложения повысит скорость обработки заказов, упростит формирование документов. 

\subsection{Веб-приложение: цели и особенности}
Платформа решает следующие проблемы:

Для клиентов:
\begin{itemize}
	\item самостоятельный подбор компонентов через удобную форму с фильтрами;
	\item доступ к базе знаний.
\end{itemize}

Для сотрудников:
\begin{itemize}
	\item централизованная система для управления заказами и клиентской базой;
	\item интеграция с поставщиками комплектующих для автоматического пополнения склада.
\end{itemize}

\subsection{Целевая аудитория}
Частные пользователи: геймеры, фрилансеры, стримеры, нуждающиеся в мощных и тихих системах.

Бизнес-клиенты: малый бизнес, требующий рабочих станций для дизайна, программирования или обработки данных.













