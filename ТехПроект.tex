\section{Технический проект}
\subsection{Общая характеристика организации решения задачи}

Необходимо разработать веб-приложение, которое организует работу в сервис-центре и поможет с его продвижением. Веб-приложение должно поспособствовать в организации работы сервиса Pc-Club и привлечь новых клиентов.

\subsection{Обоснование выбора технологии проектирования}
\subsubsection{Описание используемых технологий и языков программирования}

В процессе разработки веб-приложения потребовались языки для реализации разметки на странице, логики страницы, для настройки внешнего вида, также требуется создать базу данных для хранения всей информации. 

\paragraph{Язык разметки html}

В современном мире html является незаменимым инструментом при разработке любых веб-сайтов ведь позволяет достаточно удобно и структурировано располагать элементы на странице, добавлять таблицы и некоторое другое наполнение. Обычно на странице есть разделение на head, body, где в первом хранятся метаданные страницы, описание, ключевые слова итп., а во втором видимое содержимое страницы.

\paragraph{Язык программирования PHP}

PHP популярный серверный язык программирования, был разработан специально для веб-разработки, он встраивается в html. Главное отличие PHP от Js в том, что он исполняется на стороне сервера. Сам по себе получил широкое применение в создании серверной логики веб-сайтов/приложений, используется для взаимодействия с базами данных, обработки ввода пользователя и создания динамического контента.

\paragraph{Язык программирования JavaScript}

Js или же JavaScript ключевой язык программирования в современном frontend и важный инструмент в веб-разработке. Код выполняется в браузере и манипулирует DOM-деревом страницы, без нужды её перезагружать, позволяет отслеживать нажатия клавиш или клики мышкой в реальном времени.

\paragraph{Язык программирования CSS}

Это формальный язык описания внешнего вида страницы, определяющий стиль и расположение элементов на веб-странице. С помощью него можно тонко настраивать внешний вид элементов на странице, их расположение относительно друг друга. Когда страница создается браузер парсит html в DOM-дерево, а css в CSSOM-дерево, а после рендерит в общее дерево и отрисовывает готовую страницу.

\paragraph{База данных и Xampp}

Кросплатформенный  дистрибутив для сборки локального веб-сервера, содержит в себе Apache, MySQL нужные для реализации базы данных в разработке. Управление базой данных происходит через PhpMyAdmin - веб-приложение для администрирования БД, с помощью удобного графического интерфейса можно создавать, редактировать, а также установить на сервер. Взаимодействие с SQL происходит посредством специальных SQL запросов.

\subsection{Диаграмма компонентов и схема обмена данными между файлами компонента}

На рисунке \ref{thatsmedio1:image} изображена диаграмма компонентов для проектируемой системы. В диаграмме изображено отношение между компонентами и их взаимодействие.

\begin{figure}[ht]
\center{\includegraphics[width=1\linewidth]{thatsmedio1}}
\caption{Диаграмма компонентов}
\label{thatsmedio1:image}
\end{figure}

Любой компонент должен быть вызван в сценарии страницы web-сайта. Web-страница передает данные компоненту в момент вызова последнего.

Клиент взаимодействует с веб-сервером Apache который обрабатывает все php сценарии. Первой в очереди идет страница индекса, с нее можно попасть в orderform которая после проверки данных с помощью валидатора отправляет их в sql. Через index можно попасть в login где, происходит авторизация пользователя и если тот админ, то может попасть в панель. Панель позволяет редактировать, удалять, добавлять записи в sql, а так-же просматривать её содержимое. В случае редактирования/добавления проверяется через валидатор, перед отправкой в sql, для поддержания стабильности работы веб-приложения. В склад можно попасть с панели и редактировать количество имеющихся компонентов, получая и отправляя данные в sql.

\subsection{Структура базы данных sql и таблиц}

Реляционная схема (рис.~\ref{struct:image}) отражает отношение данных в sql в удобном табличном виде.

\vspace{-8mm} 
\begin{figure}[ht]
\center{\includegraphics[width=1.00\linewidth]{struct}}
\caption{Реляционная схема базы данных}
\label{struct:image}
\end{figure}

Все таблицы сходятся к одной таблице assembly, которая представляет собой сам "заказ". Также есть отдельная таблица users, нужна она для хранения данных пользователей.
Далее представлены все данные в табличном виде:

\renewcommand{\arraystretch}{0.8}

% Таблица Customer
\begin{xltabular}{\textwidth}{|X|X|X|X|X|}
	\caption{Описание таблицы Customer с кратким именем CTR\label{tab:customer}}\\
	\hline
	\multicolumn{1}{|c|}{Имя таблицы} & \multicolumn{4}{c|}{Краткое имя таблицы} \\ \hline
	\multicolumn{1}{|c|}{Customer} & \multicolumn{4}{c|}{CTR} \\ \hline
	\multicolumn{1}{|c|}{Key Type} & \multicolumn{1}{c|}{Optionality} & \multicolumn{1}{c|}{Column name} & \multicolumn{1}{c|}{Data type} & \multicolumn{1}{c|}{Size} \\ \hline
	pk & * & ctr\_id & NUMBER & 13 \\ \hline
	& * & full\_name & VARCHAR2 & 110 \\ \hline
	& * & phone\_namber & NUMBER & 11 \\ \hline
	& * & legal\_address & VARCHAR2 & 110 \\ \hline
\end{xltabular}

% Таблица Master
\begin{xltabular}{\textwidth}{|X|X|X|X|X|}
	\caption{Описание таблицы Master с кратким именем MTR\label{tab:master}}\\
	\hline
	\multicolumn{1}{|c|}{Имя таблицы} & \multicolumn{4}{c|}{Краткое имя таблицы} \\ \hline
	\multicolumn{1}{|c|}{Master} & \multicolumn{4}{c|}{MTR} \\ \hline
	Key Type & Optionality & Column name & Data type & Size \\ \hline
	pk & * & mtr\_id & NUMBER & 13 \\ \hline
	& * & full\_name & VARCHAR2 & 110 \\ \hline
	& * & phone\_namber & NUMBER & 11 \\ \hline
	& * & legal\_address & VARCHAR2 & 110 \\ \hline
	& * & legal\_address & VARCHAR2 & 110 \\ \hline
\end{xltabular}

% Таблица Case
\begin{xltabular}{\textwidth}{|X|X|X|X|X|}
	\caption{Описание таблицы Mcase с кратким именем CSE\label{tab:case}}\\
	\hline
	\multicolumn{1}{|c|}{Имя таблицы} & \multicolumn{4}{c|}{Краткое имя таблицы} \\ \hline
	\multicolumn{1}{|c|}{Mcase} & \multicolumn{4}{c|}{CSE} \\ \hline
	Key Type & Optionality & Column name & Data type & Size \\ \hline
	pk & * & cse\_id & NUMBER & 13 \\ \hline
	& * & case\_name & VARCHAR2 & 50 \\ \hline
	& * & form\_factor & VARCHAR2 & 20 \\ \hline
	& * & case\_size & VARCHAR2 & 30 \\ \hline
	& * & case\_manufacturer & VARCHAR2 & 110 \\ \hline
	& * & price & NUMBER & 30 \\ \hline
	& * & stock & NUMBER & 5 \\ \hline
\end{xltabular}

% Таблица Motherboard
\begin{xltabular}{\textwidth}{|X|X|X|X|X|}
	\caption{Описание таблицы Motherboard с кратким именем MBD\label{tab:motherboard}}\\
	\hline
	\multicolumn{1}{|c|}{Имя таблицы} & \multicolumn{4}{c|}{Краткое имя таблицы} \\ \hline
	\multicolumn{1}{|c|}{Motherboard} & \multicolumn{4}{c|}{MBD} \\ \hline
	Key Type & Optionality & Column name & Data type & Size \\ \hline
	pk & * & mbd\_id & NUMBER & 13 \\ \hline
	& * & motherboard\_name & VARCHAR2 & 50 \\ \hline
	& * & form\_factor & VARCHAR2 & 20 \\ \hline
	& * & chipset & VARCHAR2 & 30 \\ \hline
	& * & socket & VARCHAR2 & 30 \\ \hline
	& * & board\_manufacturer & VARCHAR2 & 110 \\ \hline
	& * & price & NUMBER & 30 \\ \hline
	& * & stock & NUMBER & 5 \\ \hline
\end{xltabular}

% Таблица Processor
\begin{xltabular}{\textwidth}{|X|X|X|X|X|}
	\caption{Описание таблицы Processor с кратким именем CPU\label{tab:processor}}\\
	\hline
	\multicolumn{1}{|c|}{Имя таблицы} & \multicolumn{4}{c|}{Краткое имя таблицы} \\ \hline
	\multicolumn{1}{|c|}{Processor} & \multicolumn{4}{c|}{CPU} \\ \hline
	Key Type & Optionality & Column name & Data type & Size \\ \hline
	pk & * & cpu\_id & NUMBER & 13 \\ \hline
	& * & unit\_name & VARCHAR2 & 50 \\ \hline
	& * & socket & VARCHAR2 & 30 \\ \hline
	& * & base\_frequency & NUMBER & 20 \\ \hline
	& * & number\_of\_cores & NUMBER & 10 \\ \hline
	& * & cpu\_manufacturer & VARCHAR2 & 110 \\ \hline
	& * & price & NUMBER & 30 \\ \hline
	& * & stock & NUMBER & 5 \\ \hline
\end{xltabular}

% Таблица RAM
\begin{xltabular}{\textwidth}{|X|X|X|X|X|}
	\caption{Описание таблицы RAM с кратким именем RAM\label{tab:ram}}\\
	\hline
	\multicolumn{1}{|c|}{Имя таблицы} & \multicolumn{4}{c|}{Краткое имя таблицы} \\ \hline
	\multicolumn{1}{|c|}{RAM} & \multicolumn{4}{c|}{RAM} \\ \hline
	Key Type & Optionality & Column name & Data type & Size \\ \hline
	pk & * & ram\_id & NUMBER & 13 \\ \hline
	& * & ram\_name & VARCHAR2 & 50 \\ \hline
	& * & memory\_size & NUMBER & 30 \\ \hline
	& * & type & VARCHAR2 & 10 \\ \hline
	& * & base\_frequency & NUMBER & 20 \\ \hline
	& * & ram\_manufacturer & VARCHAR2 & 110 \\ \hline
	& * & price & NUMBER & 30 \\ \hline
	& * & stock & NUMBER & 5 \\ \hline
\end{xltabular}

% Таблица Storage
\begin{xltabular}{\textwidth}{|X|X|X|X|X|}
	\caption{Описание таблицы Storage с кратким именем SDU\label{tab:storage}}\\
	\hline
	\multicolumn{1}{|c|}{Имя таблицы} & \multicolumn{4}{c|}{Краткое имя таблицы} \\ \hline
	\multicolumn{1}{|c|}{Storage} & \multicolumn{4}{c|}{SDU} \\ \hline
	Key Type & Optionality & Column name & Data type & Size \\ \hline
	pk & * & sdu\_id & NUMBER & 13 \\ \hline
	& * & storage\_name & VARCHAR2 & 50 \\ \hline
	& * & storage\_capacity & NUMBER & 20 \\ \hline
	& * & reading\_speed & NUMBER & 20 \\ \hline
	& * & sdu\_type & VARCHAR2 & 20 \\ \hline
	& * & sdu\_manufacturer & VARCHAR2 & 110 \\ \hline
	& * & price & NUMBER & 30 \\ \hline
	& * & stock & NUMBER & 5 \\ \hline
\end{xltabular}

% Таблица Power Unit
\begin{xltabular}{\textwidth}{|X|X|X|X|X|}
	\caption{Описание таблицы Power\_unit с кратким именем PSU\label{tab:psu}}\\
	\hline
	\multicolumn{1}{|c|}{Имя таблицы} & \multicolumn{4}{c|}{Краткое имя таблицы} \\ \hline
	\multicolumn{1}{|c|}{Power\_unit} & \multicolumn{4}{c|}{PSU} \\ \hline
	Key Type & Optionality & Column name & Data type & Size \\ \hline
	pk & * & psu\_id & NUMBER & 13 \\ \hline
	& * & power\_name & VARCHAR2 & 50 \\ \hline
	& * & capability & NUMBER & 20 \\ \hline
	& * & power\_manufacturer & VARCHAR2 & 110 \\ \hline
	& * & price & NUMBER & 30 \\ \hline
	& * & stock & NUMBER & 5 \\ \hline
\end{xltabular}

% Таблица Graphics Card
\begin{xltabular}{\textwidth}{|X|X|X|X|X|}
	\caption{Описание таблицы Graphics\_card с кратким именем GPU\label{tab:gpu}}\\
	\hline
	\multicolumn{1}{|c|}{Имя таблицы} & \multicolumn{4}{c|}{Краткое имя таблицы} \\ \hline
	\multicolumn{1}{|c|}{Graphics\_card} & \multicolumn{4}{c|}{GPU} \\ \hline
	Key Type & Optionality & Column name & Data type & Size \\ \hline
	pk & * & gpu\_id & NUMBER & 13 \\ \hline
	& * & gpu\_name & VARCHAR2 & 50 \\ \hline
	& * & gmemory\_size & NUMBER & 20 \\ \hline
	& * & gpu\_series & VARCHAR2 & 30 \\ \hline
	& * & gpu\_manufacturer & VARCHAR2 & 110 \\ \hline
	& * & price & NUMBER & 30 \\ \hline
	& * & stock & NUMBER & 5 \\ \hline
\end{xltabular}

% Таблица Assembly Order
\begin{xltabular}{\textwidth}{|X|X|X|X|X|}
	\caption{Описание таблицы Assembly\_order с кратким именем ASO\label{tab:aso}}\\
	\hline
	\multicolumn{1}{|c|}{Имя таблицы} & \multicolumn{4}{c|}{Краткое имя таблицы} \\ \hline
	\multicolumn{1}{|c|}{Assembly\_order} & \multicolumn{4}{c|}{ASO} \\ \hline
	Key Type & Optionality & Column name & Data type & Size \\ \hline
	pk & * & assembly\_order\_id & NUMBER & 13 \\ \hline
	& * & assembly\_price & NUMBER & 50 \\ \hline
	& * & date\_of\_admission & DATE & 10 \\ \hline
	& * & date\_of\_delivery & DATE & 10 \\ \hline
	& * & delivery\_address & VARCHAR2 & 110 \\ \hline
	fk & * & ctr\_id & NUMBER & 13 \\ \hline
	fk & * & mtr\_id & NUMBER & 13 \\ \hline
	fk & * & case\_id & NUMBER & 13 \\ \hline
	fk & * & mbd\_id & NUMBER & 13 \\ \hline
	fk & * & cpu\_id & NUMBER & 13 \\ \hline
	fk & * & ram\_id & NUMBER & 13 \\ \hline
	fk & * & sdu\_id & NUMBER & 13 \\ \hline
	fk & * & psu\_id & NUMBER & 13 \\ \hline
	fk & * & gpu\_id & NUMBER & 13 \\ \hline
\end{xltabular}

% Таблица Users
\begin{xltabular}{\textwidth}{|X|X|X|X|X|}
	\caption{Описание таблицы Users с кратким именем USR\label{tab:users}}\\
	\hline
	\multicolumn{1}{|c|}{Имя таблицы} & \multicolumn{4}{c|}{Краткое имя таблицы} \\ \hline
	\multicolumn{1}{|c|}{Users} & \multicolumn{4}{c|}{USR} \\ \hline
	Key Type & Optionality & Column name & Data type & Size \\ \hline
	pk & * & id & NUMBER & 11 \\ \hline
	& * & login & VARCHAR & 50 \\ \hline
	& * & password & VARCHAR & 255 \\ \hline
	& * & adminflag & NUMBER & 1 \\ \hline
\end{xltabular}

\renewcommand{\arraystretch}{1.0}


