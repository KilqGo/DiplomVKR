\section*{ОБОЗНАЧЕНИЯ И СОКРАЩЕНИЯ}

БД -- база данных.

ИС -- информационная система.

ИТ -- информационные технологии. 

ЯП -- язык программирования.

ПО -- программное обеспечение.

ЭВМ -- Электронно-вычислительная машина.

ПК -- персональный компьютер.

РП -- рабочий проект.

СУБД -- система управления базами данных.

ТЗ -- техническое задание.

ТП -- технический проект.

HTML -- стандартизированный язык разметки для создания веб-страниц и других типов цифрового контента, который может отображаться в интернете.

PHP --  это скриптовый язык программирования с открытым исходным кодом. Изначально создавался для разработки веб-приложений, но в процессе обновлений стал языком общего назначения.

CSS -- это формальный язык описания внешнего вида страницы, определяющий стиль и расположение элементов на веб-странице.
