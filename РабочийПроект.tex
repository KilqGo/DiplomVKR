\section{Рабочий проект}
\subsection{Спецификация компонентов и классов системы}
Можно выделить следующий список классов и их методов, использованных при разработке веб-приложения (таблица \ref{class:table}).

\renewcommand{\arraystretch}{0.8} % уменьшение расстояний до сетки таблицы
\begin{xltabular}{\textwidth}{|X|p{2.5cm}|>{\setlength{\baselineskip}{0.7\baselineskip}}p{4.85cm}|>{\setlength{\baselineskip}{0.7\baselineskip}}p{4.85cm}|}
\caption{Описание классов и компонентов используемых в приложении\label{class:table}}\\
\hline \centrow \setlength{\baselineskip}{0.7\baselineskip} Название класса & \centrow \setlength{\baselineskip}{0.7\baselineskip} Модуль, к которому относится класс & \centrow Описание класса & \centrow Методы \\
\hline \centrow 1 & \centrow 2 & \centrow 3 & \centrow 4\\ \hline
\endfirsthead
\caption*{Продолжение таблицы \ref{class:table}}\\
\hline \centrow 1 & \centrow 2 & \centrow 3 & \centrow 4\\ \hline
\finishhead
Validator & Валидатор & Validator - реализует логику валидации и проверки данных, которые намереваются отправить в sql, для поддержания её стабильности & required - проверка наличия значения, type - numeric, date, string, max - ограничение длины строки, pattern - валидация email/телефона, foreign - cуществование в связанной таблице.\\\hline
 * & * & * & * \\\hline 
 * & * & * & * 
\end{xltabular}
\renewcommand{\arraystretch}{1.0} % восстановление сетки

\newpage 
\subsection{Описание элементов интерфейса пользователя}

На рисунке \ref{index:image} главная страница сайта Pc-Club содержит информацию, навигацию, пример работ.

\begin{enumerate}
	\item Панель навигации.
	\item Переход на страницу админ панели.
	\item Переход на страницу оформления.
	\item Смена учетной записи.
	\item Информационный блок.
	\item Переход к оформлению.
	\item Информационный блок, преимущества Pc-Club.
	\item Блок с примерами работ.
	\item Кастомный скролл.
	\item Footer сайта с информацией.
\end{enumerate}

\begin{figure}[ht]
\center{\includegraphics[width=1\linewidth]{index}}
\caption{Главная страница сайта.}
\label{index:image}
\end{figure}

\newpage 
На рисунке \ref{main:image} представлено меню оформления заказа,  в каждом поле можно вписать вручную или выбрать из выпадающего списка.
\begin{enumerate}
	\item Переход на страницу админ панели.
	\item Переход на главную страницу.
	\item Смена учетной записи.
	\item Простая форма ввода только чисел для цены.
	\item Форма ввода с календарем для вывода дат, проверяет на корректность.
	\item Простая форма ввода для адреса.
	\item Выбор из выпадающего списка.
	\item Выбор из выпадающего списка с возможностью ручного ввода.
	\item Кнопка собирает информацию с форм и отправляет в sql.
	\item Выпадающее меню.
	\item Поле ручного ввода/поиска по компонентам в sql.
	\item Кнопка развернуть.
	\item Виды комплектующих из sql.
	\item Сколько этих комплектующих осталось на складе.
\end{enumerate}

\begin{figure}[H]
\center{\includegraphics[width=0.9\linewidth]{order+fl}}
\center{\includegraphics[width=0.9\linewidth]{order+fln}}
\center{\includegraphics[width=0.75\linewidth]{chosenone}}
\caption{Страница оформления заказа и выпадающее меню компонента.}
\label{main:image}
\end{figure}

На рисунке \ref{login:image} меню для авторизации пользователей в системе, после авторизации под админской учетной записью можно получить доступ к админ панели, складу.
\begin{enumerate}
	\item Форма ввода логина.
	\item Форма ввода пароля.
	\item Кнопка сравнивает логин и хэш пароля с хранящимся в sql, при совпадении пропускает.
	\item Кнопка сохраняет логин и хэш пароля в sql.
	\item Вернуться на главную страницу.
\end{enumerate}
\begin{figure}[ht]
	\center{\includegraphics[width=0.25\linewidth]{login}}
	\caption{Разделы для каждого вида компонентов.}
	\label{login:image}
\end{figure}

На рисунке \ref{adminall:image} Панель администратора Pc-Club позволяет полностью контролировать и редактировать информацию в sql.

\begin{enumerate}
	\item Панель навигации по сайту.
	\item Открыть меню выбора компонента.
	\item Ручной поиск по позициям и кнопка.
	\item Добавление новой записи в sql, через админ панель можно напрямую добавить новые компоненты в sql для их дальнейшего использования.
	\item Кнопка для редактирования, можно редактировать любую запись сделаную в таблицы sql.
	\item Кнопка для удаления, удалить любую запись из sql.
	\item Кнопка для экспорта данных таблицы заказа в файл.
	\item Заголовок таблицы, в нашем случае для заказов.
	\item Контент таблицы, хранимый и извлеченный из sql.
\end{enumerate}

\begin{figure}[ht]
\center{\includegraphics[width=1\linewidth]{adminall}}
\caption{Админ-панель Pc-Club общий вид}
\label{adminall:image}
\end{figure}

На рисунке \ref{razdeli:image} меню выбора между типом компонентов в панели админа.

\begin{figure}[htbp]
\center{\includegraphics[width=0.25\linewidth]{razdeli}}
\caption{Разделы для каждого вида компонентов.}
\label{razdeli:image}
\end{figure}

На рисунке \ref{storedf:image} меню менеджмента склада для веб-приложения, здесь можно пополнять запасы недостающих комплектующих, потом применять их в заказах.

\begin{enumerate}
	\item Расширенная навигационная панель админа.
	\item Кнопка вызова меню выбора типа компонента.
	\item Ручной поиск по позициям и кнопка.
	\item Заголовок таблицы.
	\item Компонент из таблицы.
	\item Количество компонента на складе и поле для ввода нового значения.
	\item Кнопка для обновления значения.
\end{enumerate}

\begin{figure}[ht]
	\center{\includegraphics[width=1\linewidth]{storedf}}
	\caption{Страничка склада корпусов в качестве примера.}
	\label{storedf:image}
\end{figure}
\clearpage

\subsection{Тестирование программной системы}
Целью данного тестирования является проверка функциональности, надежности и производительности программно-информационной системы для управления сервис-центром.
Системное тестирование позволяет выявить и устранить ошибки, а также оценить соответствие системы предъявленным требованиям.

Все тестовые сценарии должны быть выполнены успешно без критических ошибок. Система должна обеспечивать стабильную работу и удобство использования в рамках установленных бизнес-процессов. Результаты тестирования будут использованы для повышения качества и надежности программного продукта.

\textbf{1) Запуск системы}

Описание: Система должна запускаться без ошибок и отображать главное окно интерфейса.

\begin{figure}[ht]
	\center{\includegraphics[width=1\linewidth]{index0}}
	\caption{Главная страница.}
	\label{storedf:index0}
\end{figure}

\newpage

\textbf{2) Регистрация пользователя}

Описание: Пользователь должен иметь возможность создать новую учетную запись в системе.
Ожидаемый результат:  При вводе имени пользователя и пароля, нажатии на кнопку "Регистрация" пользователь создает новую запись в SQL.

\begin{figure}[ht]
	\center{\includegraphics[width=0.9\linewidth]{login0}}
	\caption{Страница логин.}
	\label{storedf:login0}
\end{figure}

\begin{figure}[ht]
	\center{\includegraphics[width=0.5\linewidth]{loginus}}
	\caption{Регистрация.}
	\label{storedf:loginus}
\end{figure}

\begin{figure}[ht]
	\center{\includegraphics[width=0.5\linewidth]{loginc}}
	\caption{Результат регистрации.}
	\label{storedf:loginc}
\end{figure}

Новая запись добавляется в SQL таблицу users, без прав администратора. Пароль хэшируется для безопасного хранения данных пользователей.

\begin{figure}[ht]
	\center{\includegraphics[width=1\linewidth]{login1}}
	\caption{Результат регистрации в SQL.}
	\label{storedf:login1}
\end{figure}

\textbf{2) Авторизация пользователя}

Описание: Пользователь должен иметь возможность войти в свою учетную запись используя корректные имя пользователя и пароль.
Ожидаемый результат: Пользователь входит в свою учетную запись и получает доступ к своим функциям.

\begin{figure}[ht]
	\center{\includegraphics[width=0.6\linewidth]{loginad}}
	\caption{Окно авторизации.}
	\label{storedf:loginad}
\end{figure}

\begin{figure}[ht]
	\center{\includegraphics[width=0.9\linewidth]{panel0}}
	\caption{Результат авторизации пользователя.}
	\label{storedf:panel0}
\end{figure}

\newpage

\textbf{3) Оформление заказа}

Описание: Пользователь должен иметь возможность создать новый заказ в системе.
Ожидаемый результат:  При вводе всех данных корректно отправляет новый заказ в систему.

\begin{figure}[ht]
	\center{\includegraphics[width=0.9\linewidth]{form0}}
	\caption{Окно оформления заказа.}
	\label{storedf:form0}
\end{figure}

\begin{figure}[ht]
	\center{\includegraphics[width=0.5\linewidth]{form1}}
	\caption{Ввод данных заказа.}
	\label{storedf:form1}
\end{figure}

\begin{figure}[ht]
	\center{\includegraphics[width=0.5\linewidth]{form2}}
	\caption{Ввод технических данных компонентов заказа.}
	\label{storedf:form2}
\end{figure}

\begin{figure}[ht]
	\center{\includegraphics[width=0.4\linewidth]{formsearch0}}
	\caption{Селектор компонентов.}
	\label{storedf:formsearch0}
\end{figure}

\begin{figure}[ht]
	\center{\includegraphics[width=0.4\linewidth]{formsearch1}}
	\caption{Результат поиска в селекторе.}
	\label{storedf:formsearch1}
\end{figure}

\begin{figure}[ht]
	\center{\includegraphics[width=0.9\linewidth]{formgg1}}
	\caption{Результат оформления заказа в SQL.}
	\label{storedf:formgg1}
\end{figure}
\clearpage

\textbf{4) Отображение панели администратора}

Описание: Пользователь авторизован как администратор, панель администратора должна отображаться без ошибок. 

\begin{figure}[ht]
	\center{\includegraphics[width=0.9\linewidth]{panel0}}
	\caption{Отображение рабочей панели администратора.}
	\label{storedf:panel0_}
\end{figure}

\textbf{5) Редактирование данных записи}
\begin{enumerate}
\item Пользователь нажимает на кнопку "карандаш".
\item Меняет данные в заказе.
\item Сохраняет изменения.
\end{enumerate}

\begin{figure}[ht]
	\center{\includegraphics[width=0.9\linewidth]{edit1}}
	\caption{Результат редактирования.}
	\label{storedf:edit1}
\end{figure}

\begin{figure}[ht]
	\center{\includegraphics[width=0.8\linewidth]{edit0}}
	\caption{Интерфейс редактирования.}
	\label{stored:edit0}
\end{figure}

\clearpage

\textbf{6) Добавление новой записи}
\begin{enumerate}
	\item Пользователь нажимает на кнопку "добавить".
	\item Заполняет форму.
	\item Сохраняет запись.
\end{enumerate}

\begin{figure}[ht]
	\center{\includegraphics[width=0.7\linewidth]{addone}}
	\caption{Добавление новой записи.}
	\label{stored:addone}
\end{figure}

\begin{figure}[ht]
	\center{\includegraphics[width=0.8\linewidth]{addone1}}
	\caption{Новая запись в системе.}
	\label{stored:addone1}
\end{figure}

\textbf{7) Удаление записи}
\begin{enumerate}
	\item Пользователь нажимает на кнопку "удалить".
	\item Подтверждает удаление.
\end{enumerate}

\begin{figure}[ht]
	\center{\includegraphics[width=0.5\linewidth]{delete}}
	\caption{Новая запись в системе.}
	\label{stored:delete}
\end{figure}

\clearpage

\textbf{8) Экспорт заказа в файл}
\begin{enumerate}
	\item Пользователь нажимает на кнопку "экспорт".
	\item Пользователь получает файл с информацией о заказе.
\end{enumerate}

\begin{figure}[ht]
	\center{\includegraphics[width=1\linewidth]{docdoc}}
	\caption{Сгенерированный .doc файл, содержит информацию о заказе.}
	\label{stored:docdoc}
\end{figure}

\textbf{9) Редактирование количества компонентов на складе}
Описание: Пользователь должен иметь возможность редактировать количество компонентов на складе.
Ожидаемый результат: Редактирует значение stock у требуемых компонентов.

\begin{figure}[ht]
	\center{\includegraphics[width=1\linewidth]{warehouse0}}
	\caption{Окно с количеством компонентов на складе.}
	\label{stored:warehouse0}
\end{figure}

\begin{figure}[ht]
	\center{\includegraphics[width=1\linewidth]{stock0}}
	\caption{Добавляем новое количество компонента.}
	\label{stored:stock0}
\end{figure}

\begin{figure}[ht]
	\center{\includegraphics[width=1\linewidth]{stock1}}
	\caption{Уведомление об успехе.}
	\label{stored:stock1}
\end{figure}

\begin{figure}[ht]
	\center{\includegraphics[width=1\linewidth]{stock2}}
	\caption{Обновленное количество компонента.}
	\label{stored:stock2}
\end{figure}

\clearpage