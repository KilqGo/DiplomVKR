\addcontentsline{toc}{section}{СПИСОК ИСПОЛЬЗОВАННЫХ ИСТОЧНИКОВ}

\begin{thebibliography}{9}

    \bibitem{javascript} 
    Абрамов, С. А. Методы оптимизации / С. А. Абрамов. – Москва: Наука, 1978. – 272 с. – Текст : непосредственный.
    
     \bibitem{javascript} 
    Васильев, В. И. Информационные системы / В. И. Васильев. – Москва: Финансы и статистика, 2009. – 416 с. – ISBN 978-5-279-03153-4. – Текст : непосредственный.
    
    \bibitem{javascript} 
    Гамзатов, М. Г. Информационные технологии / М. Г. Гамзатов, И. Г. Ахмедов. – Москва: Юрайт, 2018. – 392 с. – ISBN 978-5-534-05640-2. – Текст : непосредственный.
    
    \bibitem{javascript} 
    Голицына, О. Л. Информационные системы / О. Л. Голицына, Н. В. Максимов, И. И. Попов. – Москва: Форум, 2008. – 496 с. – ISBN 978-5-91134-245-4. – Текст : непосредственный.
    
    \bibitem{javascript} 
    Дейт, К. Введение в системы баз данных / К. Дейт. – Москва: Вильямс, 2006. – 1328 с. – ISBN 5-8459-0788-8. – Текст : непосредственный.
    
    \bibitem{javascript} 
    Информационные технологии: учебник для вузов / под ред. Н. В. Макаровой. – Москва: Финансы и статистика, 2005. – 768 с. – ISBN 5-279-02207-8. – Текст : непосредственный.
    
    \bibitem{javascript} 
    Кнут, Д. Искусство программирования / Д. Кнут. – Москва: Вильямс, 2000. – Т. 1. – 720 с. – ISBN 5-8459-0080-8. – Текст : непосредственный.
    
    \bibitem{javascript} 
    Кузнецов, С. В. Базы данных: учебник / С. В. Кузнецов. – Москва: Академия, 2007. – 496 с. – ISBN 978-5-7695-2477-0. – Текст : непосредственный.
    
    \bibitem{javascript}
    Литвиненко, В. В. Разработка информационных систем / В. В. Литвиненко. – Москва: Юрайт, 2018. – 344 с. – ISBN 978-5-534-07421-5. – Текст : непосредственный.
    
    \bibitem{javascript}
    Маклаков, С. В. BPWin и ERWin. CASE-средства разработки информационных систем / С. В. Маклаков. – Москва: ДИАЛОГ-МИФИ, 2000. – 304 с. – ISBN 5-86404-090-3. – Текст : непосредственный.
    
    \bibitem{javascript}
    Мартин, Д. Организация баз данных в вычислительных системах / Д. Мартин. – Москва: Мир, 1980. – 662 с. – Текст : непосредственный.
    
    \bibitem{javascript}
    Новиков, Ф. А. Дискретная математика для программистов / Ф. А. Новиков. – Санкт-Петербург: Питер, 2000. – 304 с. – ISBN 5-272-00176-4. – Текст : непосредственный.
    
    \bibitem{javascript}
    Робертсон, С. Осваиваем SQL за 24 часа / С. Робертсон. – Москва: Вильямс, 2002. – 272 с. – ISBN 5-8459-0315-7. – Текст : непосредственный.
    
    \bibitem{javascript}
    Таненбаум, Э. Современные операционные системы / Э. Таненбаум. – Санкт-Петербург: Питер, 2002. – 1040 с. – ISBN 5-318-00299-4. – Текст : непосредственный.
    
    \bibitem{javascript}
    Титтел, Э. HTML5 и CSS3 для чайников / Э. Титтел, К. Минник. – Москва: Вильямс, 2016. – 400 с. – ISBN 978-1-118-65720-1. – Текст : непосредственный.
    
    \bibitem{javascript}
    Электронный ресурс]. – URL: http://www.website.com (дата обращения: 15.05.2023). – Текст : электронный.
    
    \bibitem{javascript}
    Иванов, И. И. Анализ предметной области [Электронный ресурс] / И. И. Иванов. – Режим доступа: URL: http://www.example.com/ivanov (дата обращения: 20.05.2023). – Текст : электронный.
    
    \bibitem{javascript}
    Петров, П. П. Разработка информационной системы [Электронный ресурс] / П. П. Петров. – Режим доступа: URL: http://www.example.com/petrov (дата обращения: 25.05.2023). – Текст : электронный.
    
    \bibitem{javascript}
    Сидоров, С. С. Тестирование программного обеспечения [Электронный ресурс] / С. С. Сидоров. – Режим доступа: URL: http://www.example.com/sidorov (дата обращения: 30.05.2023). – Текст : электронный.
    
    \bibitem{javascript}
    Смирнов, А. А. Введение в базы данных [Электронный ресурс] / А. А. Смирнов. – Режим доступа: URL: http://www.example.com/smirnov (дата обращения: 05.06.2023). – Текст : электронный.
\end{thebibliography}
