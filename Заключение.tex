\section*{ЗАКЛЮЧЕНИЕ}
\addcontentsline{toc}{section}{ЗАКЛЮЧЕНИЕ}

Разработанная программно-информационная система для сервисного центра «PC-Club» существенно повышает эффективность и качество обслуживания клиентов за счёт комплексной автоматизации ключевых бизнес-процессов и цифровизации взаимодействия между сотрудниками и посетителями. Внедрение системы обеспечивает удобство работы персонала, оптимизирует управление заказами и складскими запасами, а также улучшает коммуникацию с клиентами, что способствует росту числа заказчиков и укреплению конкурентных позиций компании на рынке.

В ходе работы были достигнуты следующие основные результаты:

\begin{enumerate}
	\item Проведен детальный анализ предметной области сервисного центра «PC-Club», включающий специфику ремонта персональных компьютеров, управление складом комплектующих и взаимодействие с клиентами. Для определения требований к системе изучены и проанализированы семь существующих программных продуктов, таких как RemOnline и HelloClient, выявлены их преимущества и недостатки.
	
	\item Разработана концептуальная модель программно-информационной системы, включающая архитектуру веб-приложения, построенную на основе UML-диаграмм прецедентов и компонентов, а также реляционную модель базы данных. Определены основные сущности системы: клиенты, заказы, комплектующие ПК, а также пользователи с разграничением ролей (администратор, клиент).
	
	\item Осуществлено проектирование трёхуровневой архитектуры системы с использованием технологий HTML, PHP и MySQL. Разработан удобный пользовательский интерфейс, включающий пять ключевых страниц: главная, оформление заказов, авторизация, админ-панель и управление складом. Внедрены механизмы безопасности, такие как хеширование паролей, защита от SQL-инъекций с помощью подготовленных запросов и валидация вводимых данных.
	
	\item Реализована и протестирована программно-информационная система с функционалом CRUD-операций для заказов и комплектующих, возможностью отслеживания остатков на складе в реальном времени, экспортом заказов в .doc-файлы и разделением прав доступа между администраторами и клиентами. Проведено системное тестирование, подтвердившее полное соответствие требованиям технического задания.
	
\end{enumerate}

Все требования, объявленные в техническом задании, были полностью реализованы, все задачи, поставленные в начале разработки проекта, были также решены.

Готовый рабочий проект представляет собой программно-информационную систему из серверной части, SQL и веб-приложения.