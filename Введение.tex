\section*{ВВЕДЕНИЕ}
\addcontentsline{toc}{section}{ВВЕДЕНИЕ}

Пользователи персональных компьютеров периодически сталкиваются с необходимостью их ремонта или обслуживания. Ремонтом и техническим обслуживанием  компьютеров занимаются сервисные центры к которым относится «Pc-Club». Сервис-центры играют важную роль в качестве связующего звена между производителями и клиентами, обеспечивая качественную сборку, поддержку исправности техники, ремонт в случае каких либо неисправностей, настройку и консультации для конечных покупателей. В условиях высокой конкуренции и стремительного развития цифровых технологий, создание удобной и практичной системы оптимизирует часть ручной работы. Такое приложение не только позволяет повысить скорость, качество обслуживания, но и увеличить число клиентов, улучшить коммуникацию и оптимизировать внутренние бизнес-процессы.

Сервис-центр проводит ремонт либо в своих мастерских, либо возможен выезд мастера непосредственно на место оказания услуги. В сами же услуги входит ремонт материнской платы, чистка от пыли, замена АКБ, матрицы и куллера ноутбуков, ремонт цепей питания. Также сервис-центр занимается восстановлением данных с поврежденных носителей информации. Кроме всего перечисленного они занимаются администрированием компьютеров и ремонтом периферии к ним, в список услуг также входит консультация по вопросам безопасности, организация локальных сетей и поддержка клиентов по любым вопросам. На все проделанные работы предоставляется гарантия действующая в течении оговоренного срока.

В сервис-центре работают квалифицированные специалисты: инженеры по ремонту, системные администраторы, консультанты по продажам и менеджеры по работе с клиентами. Такой коллектив обеспечивает комплексный подход к обслуживанию клиентов и высокое качество сервиса.

Основная задача разработки в том, чтобы увеличить количество посетителей и заказчиков, предоставить удобную систему как для работы сотрудников, так и для посетителей. Если достаточно заинтересовать пользователя, тот с большим шансом решит воспользоваться услугами нашего сервис-центра.

Для бесперебойной работы сервис-центра на складе постоянно поддерживается необходимый запас комплектующих и расходных материалов. В ассортименте склада:
\begin{itemize}
\item оригинальные запчасти от производителей;
\item универсальные комплектующие для ремонта;
\item инструменты и расходные материалы.
\end{itemize}
Поставка комплектующих осуществляется от проверенных поставщиков с регулярным обновлением ассортимента. 

Управление складом и бизнес-процессами
Для обеспечения ритмичной и эффективной работы сервис-центра на складе должны быть:
\begin{itemize}
\item точные данные об остатках и движении комплектующих;
\item организованное адресное хранение с учетом особенностей товаров;
\item быстрая обработка поступлений и отгрузок.
\end{itemize}

\emph{Цель настоящей работы} – разработка веб-приложения сервис-центра для эффективного управления, увеличения заказов, рекламы продукции и услуг. Для достижения поставленной цели необходимо решить \emph{следующие задачи:}
\begin{itemize}
\item провести анализ предметной области;
\item разработать концептуальную модель веб-приложения;
\item спроектировать веб-приложение;
\item реализовать веб-приложение.
\end{itemize}

\emph{Структура и объем работы.} Отчет состоит из введения, 4 разделов основной части, заключения, списка использованных источников, 2 приложений. Текст выпускной квалификационной работы равен \formbytotal{lastpage}{страниц}{е}{ам}{ам}.

\emph{Во введении} сформулирована цель работы, поставлены задачи разработки, описана структура работы, приведено краткое содержание каждого из разделов.

\emph{В первом разделе} на стадии описания технической характеристики предметной области приводится сбор информации о деятельности сервис-центра, для которой осуществляется разработка сайта.

\emph{Во втором разделе} на стадии технического задания приводятся требования к разрабатываемому сайту.

\emph{В третьем разделе} на стадии технического проектирования представлены проектные решения для веб-приложения.

\emph{В четвертом разделе} приводится список классов и их методов, использованных при разработке сайта, производится тестирование разработанного сайта.

В заключении излагаются основные результаты работы, полученные в ходе разработки.

В приложении А представлен графический материал.
В приложении Б представлены фрагменты исходного кода. 
